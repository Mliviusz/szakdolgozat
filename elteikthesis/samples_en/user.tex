\chapter{User documentation}
\label{ch:user}
The ability to improve the scalability, agility, and dependability of applications in cloud environments has led to a rise in the popularity of cloud-native development in recent years. The foundation of this strategy is containerization, which offers a standardized and portable way to package and distribute software.

The industry standard for scaling up containerized application management is Kubernetes, a well-known container orchestration technology. In addition to tools for automating these procedures, it offers a complete set of primitives for deploying, scaling, and managing containerized applications.

For automated web browser testing of online applications, particularly in continuous integration and continuous delivery (CI/CD) pipelines, Selenium, an open-source testing framework, is crucial. In a cloud-native setting, Selenium testing integration with Kubernetes is essential.

This project implements a Go-based Kubernetes operator, a custom controller that extends the Kubernetes API and automates the deployment and management of Selenium tests in a Kubernetes cluster while making it easy to integrate into industry-standard monitoring and alerting chains.

This documentation will cover all operator usage aspects, including installation, configuration, and customization. We will also provide examples of how to use the operator to automate the execution of Selenium tests and how to integrate it with other Kubernetes-native tools and services.

\section{Prerequisites of employing the operator}

The operator is designed to be utilized in Kubernetes clusters of business scale, whether they are hosted on public cloud servers or on-premise servers. Although the operator was designed with performance in mind, neither Kubernetes, Selenium hubs nor Prometheus monitoring stacks are intended for local development; however, they are all necessary to show the project's feasibility. Since this tutorial will show it on a local Kubernetes cluster, be aware that hardware-related issues could occur.

Requirements are the followings:
\begin{itemize}
	\item Kubernetes cluster (minikube in the manual)
	\item Shell environment (Linux/Mac OS or Windows with WSL)
	\item kubectl CLI tool
	\item make CLI tool
	\item selenium test exported to .side format
	\item rented or self-managed Selenium Hub service (moon in the manual) 
	\item operator-managed Prometheus in the Kubernetes cluster (kube-prometheus in the manual)
	\item Helm CLI tool for deploying all the above
\end{itemize}

This manual will demonstrate setting up the following Kubernetes environment before deploying the first automated test:

\begin{figure}[H]
	\centering
	\includegraphics[width=1\textwidth]{before_tests}
	\label{fig:before_tests}
	\caption{Architecture without running tests}
\end{figure}

\subsection{Installing a Minikube cluster}

Before setting up the tools required for running Selenium tests with Kubernetes, a few prerequisites need to be installed. These prerequisites include a Kubernetes cluster, a virtualization tool for Minikube, a shell environment, the kubectl CLI tool, the make CLI tool, and a Selenium test exported to .side format. In this guide, we'll go through each of these prerequisites in detail and show you how to install them on your machine. Once these prerequisites are in place, we can move on to setting up the SeleniumTest Operator and creating custom resources for running Selenium tests.

Before everything else, Windows users has to install WSL by opening the Microsoft Store on Windows and search for "Ubuntu" or any other Linux distribution of choice and launch it.

From now on, shell environment will be used universally between operating systems, meaning shell on either Linux, Mac OS or Windows' WSL.

\begin{itemize}
	\item First of all, intall kubectl:

	\begin{itemize}
		\item In the shell terminal, run the following commands:	
		\lstset{caption={Installing kubectl}}
		\begin{lstlisting}[language=bash]
			curl -LO "https://storage.googleapis.com/kubernetes-release/release/$(curl -s https://storage.googleapis.com/kubernetes-release/release/stable.txt)/bin/linux/amd64/kubectl"
			chmod +x ./kubectl
			sudo mv ./kubectl /usr/local/bin/kubectl
		\end{lstlisting}

		\item Verify that kubectl is installed:
		\lstset{caption={Verifying kubectl}}
		\begin{lstlisting}[language=bash]
			kubectl version --short
		\end{lstlisting}
	\end{itemize}

	\item Secondly, to install make, run the following command in the shell terminal:
	\lstset{caption={Installing make}}
	\begin{lstlisting}[language=bash]
		sudo apt-get install build-essential
	\end{lstlisting}
	
	\item Thirdly, to install helm, run In the shell terminal:
	\lstset{caption={Installing helm}}
	\begin{lstlisting}[language=bash]
		curl -fsSL -o get_helm.sh https://raw.githubusercontent.com/helm/helm/main/scripts/get-helm-3
		chmod 700 get_helm.sh
		./get_helm.sh
	\end{lstlisting}

	\item Fourthly, to install and configure Minikube:

	\begin{itemize}
		\item Windows users:	
		\begin{itemize}
			\item Open your web browser and go to the Minikube releases page on GitHub (https://github.com/kubernetes/minikube/releases).
			\item Find the latest release for Windows and download the executable file (e.g., minikube-windows-amd64.exe).
			\item Move the downloaded executable to a folder that is included in your PATH environment variable. For example, you can move it to the "C:\\Windows\\System32" folder.
			\item Install Docker desktop for Minikube from its offical site: https://www.docker.com/products/docker-desktop/
		\end{itemize}
		\item Linux users:
		\lstset{caption={Installing minikube cli for Linux}}
		\begin{lstlisting}[language=bash]
			curl -LO https://storage.googleapis.com/minikube/releases/latest/minikube-linux-amd64
			sudo install minikube-linux-amd64 /usr/local/bin/minikube
		\end{lstlisting}
		\item Mac users:
		\begin{itemize}
			\item Install Homebrew by running the following command in Terminal:
			\lstset{caption={Installing homebrew cli for Mac}}
			\begin{lstlisting}[language=bash]
				/bin/bash -c "$(curl -fsSL https://raw.githubusercontent.com/Homebrew/install/HEAD/install.sh)"
			\end{lstlisting}

			\item Once Homebrew is installed, you can install minikube by running the following command in Terminal:
			\lstset{caption={Installing minikube cli for Mac}}
			\begin{lstlisting}[language=bash]
				brew install minikube
			\end{lstlisting}
		\end{itemize}
		\item After the installation is complete, validate it with the following command:
		\lstset{caption={Verifying minikube cli tool}}
		\begin{lstlisting}[language=bash]
			minikube version
		\end{lstlisting}	
	\end{itemize}

	\item Lastly, we need to start Minikube:
	\begin{itemize}
		\item Minikube startup under Windows:
		\lstset{caption={Starting minikube for Windows}}
		\begin{lstlisting}[language=bash]
			minikube start --cpus=4 --memory=8G --disk-size=30G --driver=docker
		\end{lstlisting}
		\item Minikube startup under MacOS:
		\lstset{caption={Starting minikube for MacOS}}
		\begin{lstlisting}[language=bash]
			minikube start --cpus=4 --memory=8G --disk-size=20G --driver=hyperkit
		\end{lstlisting}
		\item Minikube startup under Linux
		\lstset{caption={Starting minikube for Linux}}
		\begin{lstlisting}[language=bash]
			minikube start --cpus=4 --memory=8G --disk-size=20G --driver kvm2
		\end{lstlisting}
	\end{itemize}
	\item Even though Moon advises against using Docker as a driver, it is necessary from a networking perspective for Windows users in order to be able to access the cluster from WSL. Providing Minikube with more resources is strongly recommended if possible, but these should be the absolute minimum.
	\item Verify that Minikube is running:
	\lstset{caption={Verifying minikube cluster}}
	\begin{lstlisting}[language=bash]
		kubectl cluster-info
	\end{lstlisting}
\end{itemize}

\subsection{Solving kubectl connection problem on Windows}

A configuration file for the kubectl command-line tool for Kubernetes can be found at 
\begin{lstlisting}[language=bash] 
~/.kube/config
\end{lstlisting}
In addition to the credentials required for cluster authentication, it indicates the Kubernetes cluster with which kubectl should communicate.

Depending on the setup, when installing Minikube, it creates a Kubernetes cluster in a virtual machine or a container. Minikube changes the computer's
\begin{lstlisting}[language=bash]
~/.kube/config
\end{lstlisting}
file to set up kubectl to use this new cluster automatically. As a result, kubectl can communicate with the Minikube cluster.

Windows users encounter a problem since Minikube sets the configuration file at the Windows operating system level, whereas kubectl refers to the configuration file contained in the WSL distribution.

\begin{figure}[H]
	\centering
	\includegraphics[width=1\textwidth]{window_virtualisation}
	\label{fig:window_virtualisation}
	\caption{Virtualization difference between WSL and Unix systems}
\end{figure}

We must manually generate the kubeconfig file based on the one Minikube created to fix the problem. Although the two files will be very similar, there will be a few minor adjustments because one was created for Windows and the other for Linux.

This is how the new kubeconfig file should appear:
\lstset{caption={Example kubeconfig file}}
\begin{lstlisting}[language=bash]
	apiVersion: v1
	clusters:
	- cluster:
		certificate-authority: /mnt/c/Users/Your_User/.minikube/ca.crt
		extensions:
		- extension:
			last-update: Sat, 18 Mar 2023 17:26:55 CET
			provider: minikube.sigs.k8s.io
			version: v1.27.0
		  name: cluster_info
		server: https://127.0.0.1:8443
	  name: minikube
	contexts:
	- context:
		cluster: minikube
		extensions:
		- extension:
			last-update: Tue, 02 May 2023 22:28:36 CEST
			provider: minikube.sigs.k8s.io
			version: v1.27.0
		  name: context_info
		namespace: default
		user: minikube
	  name: minikube
	current-context: minikube
	kind: Config
	preferences: {}
	users:
	- name: minikube
	  user:
		client-certificate: /mnt/c/Users/Your_User/.minikube/profiles/minikube/client.crt
		client-key: /mnt/c/Users/Your_User/.minikube/profiles/minikube/client.key
\end{lstlisting}

After creating the file, run the following command in the WSL terminal:
\lstset{caption={Setting Kubeconfig}}
\begin{lstlisting}[language=bash]
	export KUBECONFIG=path/to/your/kubeconfig
\end{lstlisting}

Lastly, verify that Minikube is reachable:
\lstset{caption={Validate Minikube}}
\begin{lstlisting}[language=bash]
	kubectl cluster-info
\end{lstlisting}

\subsection{Deploying the required applications to the cluster}

This section will set up the kube-prometheus, Moon, and seleniumTest operator on our Kubernetes cluster. These tools are essential for testing and monitoring our cluster to ensure it runs smoothly and dependably. For our cluster, kube-prometheus offers a reliable monitoring solution that enables us to keep an eye on critical parameters like resource utilization and application performance. While the SeleniumTest Operator automates the deployment and upkeep of those tests in our Kubernetes cluster, Moon is a Selenium Grid implementation that we will use to run our Selenium tests. 

We will only deploy the Prometheus operator portion of the kube-prometheus stack because of the limited hardware resources available locally, as the other components are useless for this demonstration. Enter the operator repository to get started, then execute the following commands:
\lstset{caption={Deploying Prometheus operator}}
\begin{lstlisting}[language=bash]
	kubectl apply --server-side -f prometheus/setup
	kubectl wait --for condition=Established --all CustomResourceDefinition --namespace=monitoring
	kubectl apply -f prometheus/
\end{lstlisting}

Whenever there are issues, visit kube-prometheus's official github page for more details on how to deploy it: https://github.com/prometheus-operator/kube-prometheus

The following commands will deploy Moon, the Selenium Grid implementation, so that the tests may be run:
\lstset{caption={Installing moon}}
\begin{lstlisting}[language=bash]
	helm repo add aerokube https://charts.aerokube.com/
	helm repo update
	kubectl create namespace moon
	helm upgrade --install -f moon_values.yaml -n moon moon aerokube/moon2
\end{lstlisting}
Ingress controller is disabled in the moon_values.yaml file because we won't attempt to contact Moon from outside the cluster.

The seleniumTest operator can then be deployed; perform the following steps from the operator's repository:
\lstset{caption={Deploying of operator}}
\begin{lstlisting}[language=bash]
	make deploy IMG=quay.io/molnar_liviusz/selenium-test-operator:v0.0.24
\end{lstlisting}

Finally, we need to create and change the context to the namespace for our project, where we will launch all of the Selenium tests:
\lstset{caption={Creating testing namespace}}
\begin{lstlisting}[language=bash]
	kubectl create namespace testing-ns
	kubectl config set-context --current --namespace=testing-ns
\end{lstlisting}

Downloading Lens is advised for simpler cluster management (https://k8slens.dev/). Lens offers a straightforward and understandable graphical user interface for controlling Kubernetes clusters. Users can navigate and change resources, view their cluster's current state, and keep an eye on performance indicators in real time. Additionally, Lens supports numerous clusters and offers a single location for managing them all. Lens is a fantastic alternative for developers and system administrators who need to work with Kubernetes on a regular basis due to its user-friendly interface and extensive feature set.

The outcome of all this is that we now have our cluster configured, achieving the cluster state shown in the previous architecture diagram, with the seleniumTest operator, Prometheus, to query our operator's metrics (including the test results), and Moon deployment as a Selenium Grid implementation. The test recordings are the final component needed to deploy tests.

\subsection{Recorded selenium tests with Selenium IDE}

Selenium IDE is a browser extension that may be used to record user interactions with a web application. The user can export the recorded interactions in the ".side" (Selenium Integrated Development Environment) file format once the recording is finished. This file may then be used with Selenium WebDriver to reliably and repeatedly automate the same interactions. The ".side" file provides details on the actions taken, the web page elements that were the targets, and any additional data like input values or anticipated outcomes. Developers and testers can save time and effort while creating and maintaining automated tests by utilizing Selenium IDE to record and export interactions into a ".side" file.

To create a selenium test, open the extension > Record a new test in a new project > add a project name > add the URL where the browser should open at the beginning of the test > start recording > finish recording:

\begin{figure}[H]
	\centering
	\includegraphics[width=1\textwidth]{seleniumide1}
	\label{fig:seleniumide1}
	\caption{SeleniumIDE 1}
\end{figure}

\begin{figure}[H]
	\centering
	\includegraphics[width=1\textwidth]{seleniumide2}
	\label{fig:seleniumide2}
	\caption{SeleniumIDE 2}
\end{figure}

Replacing the actions' default target variables with a specific XPath target is advised to increase test reliability. Recommend doing this while utilizing the "XPath Helper" browser plugin and the browsers' inspect mode:

\begin{figure}[H]
	\centering
	\includegraphics[width=1\textwidth]{xpathhelper}
	\label{fig:xpathhelper}
	\caption{XPath Helper}
\end{figure}

\begin{figure}[H]
	\centering
	\includegraphics[width=1\textwidth]{seleniumide3}
	\label{fig:seleniumide3}
	\caption{SeleniumIDE 3}
\end{figure}

Several tests can be conducted within a single test project, and sophisticated Selenium Grid implementations, such as Moon, can perform these tests in a paralyzed fashion to increase efficiency and shorten the testing time. Once finished, save the project into the default .side extension.

\section{Deploying Automated Selenium Tests}

Automating the running of Selenium tests on a Kubernetes cluster requires a configuration file for the test, a ConfigMap that stores the .side file. A custom resource called SeleniumTest needs to be created to define the desired state of the test, such as the image to be used, the Selenium Grid to connect to, and the schedule for running the test. With these two components in place, Kubernetes can automatically run the Selenium test according to the specified schedule, greatly simplifying web application testing.

The configmap looks resembles the following:
\lstset{caption={Example configmap with .side test}}
\begin{lstlisting}[language={Go}]
	apiVersion: v1
	kind: ConfigMap
	metadata:
	  name: testcode
	  namespace: testing-ns
	data:
	  # file-like keys
	  exampletest.side: |
		{
		  "id": "02faada0-52b8-49fe-95cc-d403a5ef9dcd",
		  "version": "2.0",
		  "name": "Operator",
		  "url": "https://accounts.freemail.hu",
		  "tests": [{
			"id": "cf677198-9658-455c-b524-c2fa55ad59f7",
			"name": "login",
			"commands": [{
			  "id": "31de9648-b82a-4322-90da-65ee6a89a9f5",
			  .
			  .
			  .
\end{lstlisting}

The following is the syntax for the seleniumTest custom resource:
\lstset{caption={SeleniumTest CR yaml format}}
\begin{lstlisting}[language={Go}]
	apiVersion: selenium.mliviusz.com/v1
	kind: SeleniumTest
	metadata:
		labels:
			app.kubernetes.io/name: seleniumtest
			app.kubernetes.io/instance: seleniumtest-sample
			app.kubernetes.io/part-of: operator
			app.kubernetes.io/managed-by: kustomize
			app.kubernetes.io/created-by: operator
		name: seleniumtest-sample
		namespace: testing-ns
	spec:
		schedule: "*/2 * * * *"
		repository: quay.io
		image: molnar_liviusz/selenium-test-runner
		tag: v0.0.10
		configMapName: testcode
		retries: "3"
		seleniumGrid: "http://moon.moon.svc:4444/wd/hub"
\end{lstlisting}

The SeleniumTest object's desired state is specified in the spec section of the custom resource with the following variables:

\begin{itemize}
	\item repository: specifies the repository for the Docker image.
	\item image: The Docker image that will be used to execute the test is specified by this option. This image is also part of the thesis project
	\item tag: This parameter specifies the tag for the Docker image.
	\item retries: The number of times the test should be rerun if it fails is specified by this option. The test counts as a pass even if one is successful.
	\item schedule: This parameter specifies the schedule for the CronJob that runs the test.
	\item seleniumGrid: This parameter specifies the URL of the Selenium Grid for running the test.
\end{itemize}

To deploy the test, use the following commands:
\lstset{caption={Deploying example tests}}
\begin{lstlisting}[language=bash]
	kubectl apply -f testfiles/testcode-configmap.yaml
	kubectl apply -f testfiles/sample-seleniumtest.yaml
\end{lstlisting}

Following the previous steps, the test results can be viewed on Prometheus, where 0 means succes, 1 means the test failed:
\begin{figure}[H]
	\centering
	\includegraphics[width=1\textwidth]{prometheus_ui}
	\label{fig:prometheus_ui}
	\caption{Seleniumtest metric within Prometheus}
\end{figure}