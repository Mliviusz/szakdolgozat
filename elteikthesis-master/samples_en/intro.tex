\chapter{Introduction}
\label{ch:intro}

As a DevOps engineer, I have seen the value and challenges of ensuring services that are reliable, scalable, and of high quality. Businesses are quickly integrating cloud-native technologies, using microservice designs to improve reliability and scalability. Consequently, modern systems are far more complicated and have many more moving elements.

Pre-deployment unit, integration, system, acceptance testing, and live monitoring with logs, metrics, traces, and alarms ensure high quality. However, most technologies can only record particular mistakes, making them unable to quantify the entire user experience.

Selenium offers the toolkit needed to implement system and acceptability level tests by simulating user behaviour. It can only be used during the testing phase prior to deployment, and the existing live production monitoring cannot reliably identify faults amongst system components.

By developing a go-based Kubernetes operator that automates selenium test runs and aggregates the results in Prometheus format, my thesis project aims to bring selenium's capabilities to the live production cloud-native space while also offering a simple-to-use instrument to integrate selenium testing into industry-standard monitoring/alerting chains.